%% LyX 2.0.6 created this file.  For more info, see http://www.lyx.org/.
%% Do not edit unless you really know what you are doing.
\RequirePackage{fix-cm}
\documentclass[english]{article}
\usepackage[T1]{fontenc}
\usepackage[utf8]{luainputenc}
\setlength{\parskip}{\medskipamount}
\setlength{\parindent}{0pt}
\usepackage{amsmath}
\usepackage{amssymb}
\usepackage{fixltx2e}

\makeatletter
%%%%%%%%%%%%%%%%%%%%%%%%%%%%%% Textclass specific LaTeX commands.
\numberwithin{equation}{section}
\numberwithin{figure}{section}
\numberwithin{table}{section}

\makeatother

\usepackage{babel}
\begin{document}

\title{Automated Routing in Pedestrian Dynamics}


\author{Arne Graf}


\date{2015-09-01}

\maketitle

\section{Introduction}


\section{ODE based Model (dep.? -> opt.?)}


\section{Modelling}


\subsection{Eikonal Equation}

The ``Eikonal Equation'' in a domain $\Omega$, subset of $\mathbb{R}^{n}$,

\begin{align*}
\vert\nabla u(x)\vert\quad= & \quad F(x),x\in\Omega,\\
\mathrm{s.t.\qquad u|_{\partial\Omega}\quad=} & \quad0\\
\end{align*}
 

yields ``first-arrival-times'' in a spacial domain provided a target
region within the domain.


\subsection{Safe Navigation using the Floorfield}


\subsection{Distances-Field}


\subsubsection{Cost of a ``full'' preprocessing step }


\subsubsection{Distances-Field and repulsive Wall-Forces}


\subsection{Variant Model}

In the latter, a new approach in modeling is described, aiming for
the avoidance of faulty interaction of pedestrians and walls while
maintaining the positive characteristics of row-formation, stop-and-go
waves and such, like seen in experimental data. In many of the existing
models using mathematical formulations in continouus domain, agents
breach wall-surfaces and get stuck inside of walls or obstacles.

There are two mechanics used to avoid ``clipping'':
\begin{enumerate}
\item The routing of pedestirans makes use of the eikonal-equation, computed
with an inhomogenious speed-function, $s(x)$, which favours keeping
a distance to obstacles, walls and corners.
\item The distance to the closest wall of each pedestrian affects the moving
speed if and only if the agent's moving vector includes a component
geared towards the wall.
\end{enumerate}
In order to keep the model simple, repulsive wall forces as seen in
Social Force Models are omitted. An analogy to repulsive pedestrian
forces though is used to keep agents from colliding with each other.
The model differs from SFMs, as other agents effect the desired moving
direction in full, the magnitude on the other hand is effected by
other agents only to a certain degree (as described in ).

\begin{alignat*}{1}
\Delta\vec{x}\quad=\quad & \Delta t\cdot\vec{v}_{res}\\
\vec{v}_{res}\quad=\quad & \left(1-\frac{1}{2}\left[(\vec{v}_{n}\cdot(-\nabla distances)_{n})+\vert(\vec{v}_{n}\cdot(-\nabla distances)_{n})\vert\right]\right)\cdot\vec{v}_{n}\\
\vec{v}_{n}\quad=\quad & g(g(\vec{v}_{ff})+g(\sum\vec{v}_{repP,i}))\\
\vec{v}_{ff}\quad=\quad & v_{ff}(\vec{x})\\
\end{alignat*}


\begin{align*}
\Delta\vec{v}\quad= & \quad\Delta t\cdot\vec{v}_{res}\\
\vec{v}_{res}\quad= & \quad\bigg(1-\frac{1}{2}\bigg[\langle\vec{v}_{n},(-\nabla distances)_{n}\rangle+\big\vert\langle\vec{v}_{n},(-\nabla distances)_{n}\rangle\big\vert\bigg]\bigg)\cdot\vec{v}_{n}\\
\vec{v}_{n}\quad= & \quad g\big(\quad g(\vec{v}_{ff})+g(\underset{\small i}{\sum}\vec{v}_{repP,i})\quad\big)\\
\vec{v}_{ff}\quad= & \quad v_{ff}(\vec{x})
\end{align*}



\subsection{Idea of Separation of a Moving-Vector into Direction and Magnitute}


\subsubsection{no clipping}


\subsubsection{Recycling the Distances Field (neg. Gradient must be saved)}

$\vec{v}_{res}\quad=\quad\bigg(1-\frac{1}{2}\bigg[\langle\vec{v}_{n},(-\nabla distances)_{n}\rangle+\big\vert\langle\vec{v}_{n},(-\nabla distances)_{n}\rangle\big\vert\bigg]\bigg)\cdot\vec{v}_{n}$


\section{Testing}


\section{Shortcomings}


\subsection{Floorfield}


\subsubsection{Multiple Goals }

The floorfield is a usefull tool in routing of pedestirans through
any geometry. 


\subsubsection{Multiple Floors}


\paragraph{Neighboring Relations}


\subsection{Avoid Clipping}


\section{Outlook}


\subsection{Usage in JuPedSim}


\subsection{Floorfields in Triangulated Domains}


\subsection{Parallelization}


\section{Appendices}


\subsection{Classes and their Relations}


\subsection{Code Snippets}


\section{Bibliography}
\end{document}
